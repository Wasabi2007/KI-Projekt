\chapter{Elemente}

\section{Maske}
Die Maske enth�lt eine Toolbar welche am oberen Rand angebracht ist. Von Dort aus k�nnen die Aktionen wie Speichern, Laden und Erstellen f�r Behaviour Trees ausgew�hlt werden. Au�erdem befindet sich in der Leiste ein Start-Button, welcher das aktuelle Szenario startet. In der rechten, oberen Ecke ist der Hide/Show-Button daf�r verantwortlich, die Speicherverwaltung und den Editor auszublenden.
Auf der rechten Seite des Bildschirm ist ein Node-Editor, mit ein aktuell Fokusierter Knoten und dessen Unterknoten im Baum bearbeitet werden kann. Dieser l�sst sich am rechten Rand bei Nichtgebrauch verstecken.
Das Mittelfenster ist der Bereich in welchem die Roboter interagieren und auch die Behaviour Trees und diverse Auswahlfenster angezeigt werden.

\section{Editor}
Sobald ein Element im aktuell angezeigten Behaviour Tree angeklickt wird, wird im Editor der aktuelle Knoten und dessen Unterknoten angezeigt. �ber die Pfeile kann der aktuelle Knoten samt untergeordneten Elementen in der gleichen Hierarchie nach rechts oder links verschoben werden.
Des weiteren gibt es eine Suchleiste in der die Tasks f�r das Dropdown-Men� nebenan auf eine Auswahl limitiert werden k�nnen und durch Auswahl in der bereits genannten Liste als Unterknoten an das aktuelle Element angef�gt werden. Die Unterknoten k�nnen durch einen Klick auf einen Entfernen-Button wieder aus dem Baum gel�scht werden.

\section{Behaviour Tree}
Der Baum taucht in der Mitte des Bildschirms auf. Dieser l�sst sich durch das Gedr�ckthalten der Maustasten veschieben. Einzelne Elemente lassen sich durch einen kurzen Klick ansprechen. Insofern ein aktuelles Szenario abl�uft und der Behaviour Tree aktiviert ist, wird der aktuelle Task samt Pfad im Baum farblich hervorgehoben.

\section{Roboter}
Roboter sind im Sichtfeld verteilt. In jeden Roboter kann ein bereits abgespeicherter Behaviour Tree geladen werden, indem durch das klicken auf einen Roboter eine Liste mit bereits abgelegten B�umen erscheint und eine Auswahl getroffen werden kann. 

\section{Task}
Es gibt zwei Arten von Tasks. Zum einen die mit und zum anderen die ohne Parameter. 