\chapter{Motivation}

\section{Konzept}
Das Projektziel ist es, einen Ingame-Editor f�r Behaviour Trees zu entwickeln, welcher B�ume erstellen, �ndern, speichern und laden kann. Diese sind in Test-Roboter implementierbar, sodass Roboter mit unterschiedlichem Verhalten gegeneinander antreten k�nnen. Eine Eigenschaften des Editors ist, dass sich Behaviour Trees auch zur Laufzeit bearbeiten lassen und nach dem erneuten Speichern und Aktualisierung abrufbar sind. Behavior Trees lassen sich aus vorgefertigten Decorator-Elementen und Tasks zusammenstellen.

\section{Entwicklungsumgebung}
Das Projekt wird in der Spiele-Engine Unity umgesetzt, da durch die bereits vorhandenen Funktionen und das gro�e Angebot an Erweiterungen eine schnelle Entwicklung erm�glicht. Diese Entwicklungsumgebung erlaubt es beispielsweise f�r das Vorhaben vorteilhafte Plugins zu installieren, in diesem konkreten Fall "NGUI", um die Schaltfl�chen f�r die Bearbeitung der Behaviour Trees bereitzustellen, was durch Unity ohne diese Erweiterung in dieser Funktionalit�t  nur mit viel Aufwand und nicht in diesem Ausma� m�glich w�re. Au�erdem wird die Engine durch das "'Unity Serializerplugin"' erweitert, welches die Speicherverwaltung der B�ume erm�glicht. 
Der Entwicklungsfokus liegt beim Editor f�r die B�ume und erst in zweiter Linie an der Testm�glichkeit durch die Roboter.

\section{Zukunftsaussichten}
Das Projekt soll eine Basis bieten, auf der gegebenenfalls in sp�teren Projekten der Hochschule, wie zum Beispiel dem Medienprojekt oder dem interdisziplin�ren Teamprojekt, aufgebaut werden kann. Da nicht alle Punkte der Zielsetzung umgesetzt werden k�nnen, gibt es Potential den Editor in Zukunft zu erweitern und/oder das Projekt in der Testm�glichkeit zu expandieren.
