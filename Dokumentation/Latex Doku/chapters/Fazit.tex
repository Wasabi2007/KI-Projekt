\chapter{Fazit und Ausblick}

In dem Programm sind alle grundlegenden Behaviour-Knoten implementiert. Dadurch ist es m�glich komplexe Verhalten zu erstellen und zu testen. 
Der Entwickler ist nicht zwingend darauf angewiesen die Laufzeit-Editor-Funktionen zu nutzen, um einen funktionierenden Baum zu erstellen und zu nutzen.
Der Editor bietet jedoch die n�tigsten Funktionen, um B�ume w�hrend der Laufzeit zu erstellen, zu laden, zu speichern und zu editieren.
Momentan k�nnen nur im Editor erstellte B�ume au�erhalb von Unity gespeichert und geladen werden, da es keine M�glichkeit gab in Unity einen Dateimanager im Zeitrahmen zu implementieren.
Das ist eine Funktion, die in Zukunft implementiert werden sollte. 
Ein Vorteil der Unity Implementierung ist, dass B�ume, die erstellt wurden, direkt getestet werden k�nnen. Als Beispiel werden momentan 4 Roboter zur Verf�gung gestellt, denen frei Verhaltensb�ume zugewiesen werden k�nnen, um diese direkt zu testen. 
Jedoch kann der Entwickler auch eigene Tasks und B�ume bauen die nicht auf diesem Roboterbeispiel beruhen.
Erweitert wird das Programm durch vom Entwickler erstellte Scripte, die von der Klasse Task oder BehaviourNode ableiten.
Das Programm lief in vorhergehenden Tests stabil und robust.
