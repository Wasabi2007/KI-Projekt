\chapter{Problemstellung}

Hier wird i.d.R. zun�chst das generell vorliegende Problem diskutiert: Was ist zu l�sen - was gibt
es bisher an L�sungsans�tzen (prinzipiell) und warum ist es wichtig, dass man dieses Problem l�st.
Letzteres ergibt sich oftmals aus der vorliegenden Anwendungssituation: Man braucht die L�sung, um
eine bestimmte Aufgabe zu erledigen, ein System aufzubauen etc. Der Bezug auf vorhandene oder auch
bisher fehlende L�sungen begr�ndet auch die Intension und Bedeutung dieser Arbeit. Dies k�nnen
allgemeine Gesichtspunkte sein - man liefert einen Beitrag f�r ein generell erkanntes oder zu
erkennendes Problem - oder man hat eben eine spezielle Systemumgebung oder Produkt (z.B. in einer
Firma u.s.w.), woraus sich dieses noch zu l�sende Problem ergibt.

Die genaue Problematik und Randbedingungen werden dann in Kapitel \hyperref[Aufgabenstellung]{Kapitel~\ref{Aufgabenstellung}}
dargestellt.