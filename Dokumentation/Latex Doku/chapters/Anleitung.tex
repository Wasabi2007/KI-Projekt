\chapter{Anleitung}
Es gibt zwei verschiedene Modi, welche rechts oberen am Bildschirmrand gewechselt werden k�nnen. Das Programm startet im Show-Modus, was bedeutet, dass der Editor angezeigt wird. Dort k�nnen B�ume angelegt, bearbeitet und gespeichert werden. Im Hide-Modus k�nnen bereits abgespeicherte B�ume den Robotern zugewiesen werden, da der Editor versteckt wird.


Show - Editor ist eingeblendet
\section{Anlegen}
Um einen neuen Baum zu erstellen wird der New-Button gedr�ckt. Daraufhin erscheint ein Fenster mit einer Liste. Dort wird der erste Knoten des Baumes ausgew�hlt.

\section{Bearbeiten}
Am rechten Rand verbirgt sich das Bearbeitungs-Men�. Mit einem Klick auf den schmalen Balken klappt sich dieses auf.

\section{Knoten Ausw�hlen}
Mit einem Klick auf den zu bearbeitenden Knoten direkt im Baum dieser direkt ausgew�hlt und ist dadurch bearbeitbar.

\section{Knoten verschieben}
Direkt unter dem Namen des Knotens befindet sich deine Index-Position in der Reihe der Unterknoten. Mit einem Klick auf den Links- oder Rechts-Buttons wird der Knoten in der Reihe dementsprechend verschoben.

\section{Knoten l�schen}
Um einen Knoten zu l�schen wird der Vaterknoten dessen ausgew�hlt. Dort stehen die Kind-Knoten aufgelistet. Rechts neben den Namen dieser befindet sich der L�schen-Button. Durch das anklicken wird der Knoten aus der Liste der Kindknoten entfernt.

\section{Speichern}
\subsection{Neuer Baum}
Beim Speichern eines neuen Baumes wird in der Leiste am oberen Bildschirmrand der "'Save As"'-Button ausgew�hlt. Dadurch �ffnet sich ein Fenster in welchem der 

\subsection{Bereits vorhandener Baum}
Wenn ein bereits gespeicherter Baum ver�ndert wurde, kann der Fortschritt durch den Save-Button gesichert werden.

\section{Laden}
�ber den Load-Button am oberen Bildschirmrand kann aus einer Liste der gespeicherten B�ume ausgew�hlt werden um diesen weiter zu editieren. 

\section{Ablauf des Baumes verfolgen}
Um den Ablauf eines Baumes sehen zu k�nnen, muss bereits mindestens ein Roboter mit einem Behaviour Tree ausgestattet sein. Beim Klick auf den Button mit der Aufschrift ?None? �ffnet sich die Liste mit Robotern, welchen bereits ein Baum zugewiesen wurde. Sobald dort ein solcher ausgew�hlt ist, werden die anderen Funktionen des Show-Modus automatisch Deaktiviert und der zugeh�rige Baum taucht auf. Sobald das Szenario gestartet wird, werden aktuelle Pfade gr�n markiert.

Hide - Editor ist ausgeblendet
Um den Baum-Editor zu verstecken bzw. auszuschalten, wird der Hide-Button gedr�ckt.

\section{Roboter einen Baum zuweisen}
Mit dem Rechtsklick der Maus auf einen beliebigen Roboter erscheint die Liste von bereits gespeicherten B�umen. Durch die Auswahl eines Baumes wird dieser dann dem Roboter zugeordnet.

Unabh�ngig von der Sichtbarkeit des Editors

\section{Szenario Starten}
Durch das Dr�cken des Start-Buttons beginnt das Szenario. Alle zugewiesenen B�ume werden automatisch aktiviert.

\section{Szenario Zur�cksetzen}
Der Reset-Button kann jederzeit, w�hrend das Szenario l�uft, gedr�ckt werden. Alle Roboter werden auf Ihre Anfangsposition samt deren Anfangs-Attributwerten zur�ckgesetzt. Die zugewiesenen B�ume bleiben nach dem Reset erhalten.