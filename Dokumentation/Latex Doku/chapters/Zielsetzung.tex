\chapter{Zielsetzung}
Die Erwartungen an das Projekt wurden in 3 Kategorien von Zielen unterteilt. Daher werden verschiedene Priorit�ten in Abh�ngigkeit an die verbleibende Zeit des Projektes gesetzt.

\section{ Projektziele }

\subsection{Muss-Ziele}
\begin{itemize}
  \item Basisroboter(mit festen Attributwerten: Hp, Schussrate, Schaden, Munition, Schnelligkeit und entsprechenden Funktionen)
  \item Einem Objekt der Klasse "Robot" wird ein Steering-Type zugewiesen. Die Funktionen zu den einzelnen Steering-Typen sind in der Robot-Klasse direkt implementiert
  \item Grundbehaviours(Flee, Seek, Attack, Wander)
  \item Ver�nderbarkeit von Tasks mit Attributen
  \item Basiseditor(Bilden/Editieren, Speichern und Laden von Behaviour Trees)
  \item Simple Tasks f�r den Editor
  \item Anzeige des aktuellen Status des ausgew�hlten Bahaviour Trees   
\end{itemize}

\subsection{Soll-Ziele}
\begin{itemize}
\item Erweiterter Editor(Baukastensystem f�r den Bau eines Bahaviour Trees)
\item Testm�glichkeit verschiedener Behaviours durch "Roboter-Kampf" (Interaktion zwischen zwei oder mehr Roboter-Objekte)
\item Face-Only-Steering
\item Ein Paar verschiedene vorgefertigte Behaviour Trees
\item Save As-Button zum Speichern von B�umen unter neuem Namen
\item Auswahlm�glichkeit f�r Trees/Roboter um den ausgef�hrten Baum anzeigen zu lassen
\item Erweiterbarkeit des Task-Pools
\item Angriff von Robotern durch schie�en
\end{itemize}


\subsection{ Kann-Ziele}
\begin{itemize}
\item Sch�ne Grafiken
\item Aufw�ndiges Design
\item Mehr bzw. komplexere Tasks f�r vielf�ltige Behaviour Trees
\item Behaviour Trees als Teilb�ume in andere B�ume implementierbar machen
\end{itemize}