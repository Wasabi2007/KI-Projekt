\chapter{Zielsetzung}
Die Erwartungen an das Projekt wurden in 3 Kategorien von Zielen unterteilt. Daher werden verschiedene Priorit�ten in Abh�ngigkeit an die verbleibende Zeit des Projektes gesetzt.

Projektziele:

Muss-Ziele:
\begin{enumerate}
  \item Basisroboter(mit festen Attributwerten: Hp, Schussrate, Schaden, Munition, Schnelligkeit und entsprechenden Funktionen)
  \item Einem Objekt der Klasse ?Robot? wird ein Steering-Type(steeringtype) zugewiesen. Die 	Funktionen zu den einzelnen Steering-Typen sind in der Robot-Klasse direkt implementiert
  \item Grundbehaviours(Flee, Seek, Attack, Wander)
  \item Ver�nderbarkeit von Tasks mit Attributen
  \item Basiseditor(Bilden/Editieren, Speichern und Laden von Behaviour Trees)
  \item Simple Tasks f�r den Editor
  \item Anzeige des aktuellen Status des ausgew�hlten Bahaviour Trees   
\end{enumerate}

Soll-Ziele:
- Erweiterter Editor(Baukastensystem f�r den Bau eines Bahaviour Trees)[]
- Testm�glichkeit verschiedener Behaviours durch ?Roboter-Kampf? (Interaktion zwischen zwei oder mehr Roboter-Objekte)
- Face-Only-Steering
- Ein Paar verschiedene vorgefertigte Behaviour Trees
- Save As-Button zum Speichern von B�umen unter neuem Namen.
- Auswahlm�glichkeit f�r Trees/Roboter um den ausgef�hrten Baum anzeigen zu lassen
- Erweiterbarkeit des Task-Pools
- Angriff von Robotern durch schie�en

Kann-Ziele:
- Sch�ne Grafiken
- Aufw�ndiges Design
- Mehr bzw. komplexere Tasks f�r vielf�ltige Behaviour Trees
- Behaviour Trees als Teilb�ume in andere B�ume implementierbar machen